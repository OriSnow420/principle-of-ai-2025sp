\section{Homework 4}

\subsection{问题2}

令$n$维向量$Y=\left(\begin{array}{c}
    y_1 \\ \vdots \\ y_n
\end{array}\right)$, $\hat{Y}=\left(\begin{array}{c}
    \hat{y_1} \\ \vdots \\ \hat{y_n}
\end{array}\right)$, $\bar{Y}=\left(\begin{array}{c}
    \bar{y} \\ \cdots \\ \bar{y}
\end{array}\right)$. 因此要证明的结论等价于$\hat{Y}-\bar{Y}$和$\hat{Y}-Y$正交, 其
数量积应当等于$0$:
\begin{equation*}
    S=\sum_{i=1}^{n}(\hat{y}_i-\bar{y})(\hat{y}_i-y_i)=0
\end{equation*}
最小二乘要求残差的平方和
\begin{align*}
    F=\sum_{i=1}^{n}(y_i-\hat{w}x_i-b)^2
\end{align*}
最小, 因此$(\hat{w},b)$应当取得其极值点, 因此有:

\begin{equation*}
    \left\{
        \begin{array}{c}
            \displaystyle \frac{\partial F}{\partial \hat{w}}=\sum_{i=1}^{n}-2x_i(y_i-\hat{w}x_i-b)=0 \\
            \displaystyle \frac{\partial F}{\partial b}=\sum_{i=1}^{n}-2(y_i-\hat{w}x-b)=0
        \end{array}
    \right.
\end{equation*}
比较$S$和偏导数的表达式, 有$\displaystyle S=\frac{1}{2}(\hat{w}\frac{\partial F}
{\partial \hat{w}}+b\frac{\partial F}{\partial b})=0$.

\subsection{问题4}

概率之和$\displaystyle\sum_{k=1}^{K}P(Y=j)=(1/Z)e^{\beta_kx}\Rightarrow\sum_{k=1}^{K}e^{\beta_kx}=Z$, 因此
\begin{equation*}
    P(Y=k)=\frac{e^{\beta_kx}}{Z}=\frac{e^{\beta_kx}}{\sum_{j=1}^{K}e^{\beta_jx}}
\end{equation*}