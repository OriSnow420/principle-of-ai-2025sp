\section{Homework 6}

\subsection{问题2}

(1) 显然甲乙两人的分数和为$0$, 每个人的分数取值为$\{-2,-1,0,1,2\}$. 设状态$S_i$表示甲的分数为$i-2$, 则$S=\{S_0,S_1,S_2,S_3,S_4\}$为状态空间, 其中$S_0,S_4$为终止状态. 状态转移矩阵\begin{equation*}
    \bf P = \left[\begin{array}{ccccc}
        1 & 0 & 0 & 0 & 0 \\
        q & r & p & 0 & 0 \\
        0 & q & r & p & 0 \\
        0 & 0 & q & r & p \\
        0 & 0 & 0 & 0 & 1
    \end{array}\right]
\end{equation*}

(2) 从$S_3$开始, 恰好赛两局结束比赛的唯一可能是先平再甲胜, 进入状态$S_4$, 概率$P=rp$.

\subsection{问题3}

设左上和右下两个终止状态的编号为$0,8$. $q_\pi(7,\text{right})=R_t+q_\pi(8)$, 由于$8$为终止状态, 有$q_\pi(8)=0$. 因此$q_\pi(7,\text{right}=0)=-1$.

对于$q_\pi(4,\text{left})$, 有: \begin{equation*}
    \left\{\begin{array}{ll}
        q_\pi(4,\text{left}) = -1 + q_\pi(3) \\
        q_\pi(3) = -1 + \frac{1}{4}[q_\pi(0)+q_\pi(3)+q_\pi(6)+q_\pi(4)] \\
        q_\pi(6) = -1 + \frac{1}{4}[q_\pi(3)+q_\pi(7)+2q_\pi(6)]
    \end{array}\right.
\end{equation*}
由于问题的对称性, 可以得到\begin{equation*}
    \left\{
        \begin{array}{ll}
            q_\pi(4)=q_\pi(4,\text{left}) \\
            q_\pi(3)=q_\pi(7)
        \end{array}
    \right.
\end{equation*}

结合已知条件$q_\pi(0)=0$, 解方程组得 $q_\pi(4,\text{left})=-8$.